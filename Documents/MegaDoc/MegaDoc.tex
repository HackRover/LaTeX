\documentclass[a4paper, 10pt]{article}

\usepackage{graphicx} %%Good primer on this see:  https://www.youtube.com/watch?v=Ax9RCvjpI8E

\usepackage{geometry} %%To set basic page parameters. See also: https://www.overleaf.com/learn/latex/Page_size_and_margins
\geometry{a4paper, portrait, margin=1in}

%%Use this when you want to change colors eventually: https://tex.stackexchange.com/questions/75667/change-colour-on-chapter-section-headings-lyx

%%This is a function I found on stack overflow when I looked up
%%'How can I change the margins for only part of the text?'
\def\changemargin#1#2{\list{}{\rightmargin#2\leftmargin#1}\item[]} 
\let\endchangemargin=\endlist 
%%Use it sparingly.

\renewcommand \thesection {\Alph{section}} %%Defines section numberings (A.1.1, etc)

\begin{document}

\begin{titlepage}

	\begin{figure}[h]
		\centering
		\includegraphics[scale=.8]{HackRoverlogolofi}
	\end{figure}

	\begin{center}
		\vspace*{1cm}
	
		\Huge
		\textbf{HackRover}\\[10pt]
	
	
		\large
		\textbf{Capstone MegaDoc}\\[40pt]

		\begin{changemargin}{10pt}{10pt} 
		\begin{center}
		\normalsize
		Justin Heinzig, Danny Kha, Jacob Park
	
		Advisor; Mentor; Sponsor: Pierre Mourad\\[150pt]
		\end{center}
		\end{changemargin}
	\end{center}
	
	\begin{figure}[h]
		\centering
		\includegraphics[scale=.8]{UWLogo}
	\end{figure}		
	
\end{titlepage}

\pagebreak{}

\tableofcontents

\pagebreak{}

\section{Design Strategy}
	
	\subsection{User Insights and Research} 
		\subsubsection{Consumer Journey Map}
			The above figure shows a laid-out journey map that a possible consumer may go through with our HackRover product. The consumer journey starts with a disaster that is assessed to need some type of rover to assist in disaster relief. The HackRover project will be able to accomplish this by creating configurable rovers to assist disaster relief teams in quickly designing a rover to fit any type of disaster scenario that may be presented.
		\subsubsection{Stakeholder Framing}
			Stakeholder: Disaster Relief Operators: A design focus of the rovers is a disaster relief rover, though they won’t likely purchase the device themselves, being able to operate it effectively is a function of our good design. Operators will be able to quickly modify the rover to fit the disaster needs and fit a function that will be able to assist with disaster relief. 
			
Stakeholder: Government Relief Organizers: The likely organization that would be purchasing our disaster relief rover. Making a robust cost-effective system is the most important selling point. If troops or units can be deployed inexpensively and provide vital disaster data, they’re sure to sell. 

Stakeholder: UW Bothell Students: UW Students are the primary users of the product. Though they haven’t purchased any rovers or paid fees for admission into the competition, they nonetheless benefit from its free access.

Stakeholder: Pierre Mourad: Pierre derives benefits from the product in the form of reputation of his person, outreach for access to students (for other projects potentially), and access to robotics systems. Our performance in this capstone cycle will affect trust in his decisions. Beyond this, many components from the past were purchased by him in some way.

Stakeholder: UW Capstone Core Team: Like UWB Students, though we have not paid for anything large, we are nonetheless a stakeholder. Our performance on this project correlates with future successes in the industry, as this capstone is a demonstration of our ability.

Stakeholder: UW Student Technology Fund: The STF is the primary contributor to our organization, providing more than 3000 dollars in hardware technology. The exchange of this technology was meaningful use, future requests need to be paired with demonstration of this good use.

		\subsubsection{User Scenarios and Personas}
		Three Personas:
		
- Students do not have enough extracurricular engineering opportunities at UW Bothell where they can apply the knowledge they have gained from courses.

- There are cases where there are currently no robotic devices already made or ready to be configured to be sent to certain disaster areas.

- Engineering firms want students to have a more extensive education in documentation and industry-tier engineering practice.

Persona Explanation:

- UW Bothell has funding to provide extracurricular clubs which would provide students the opportunity to apply their knowledge to a club like HackRover. Students will participate in a club that provides a timeline for increasing their marketable skills. Because the job market has always been competitive, UW Bothell students need to be competitive, and having HackRover as a background of extracurricular activity can help.

-Disaster relief companies need robots that are either already made for a specific scenario or a robot that can be easily configured to be made for the scenario. Companies can fund the UWB HackRover club to provide designs and testing for these types of disaster-relief robots.

- Engineering firms would rather students have some experience which better reflects the industry, prior to entering the job market. HackRover can provide the industry experience that engineering firms are looking for from graduates. 

Pain Points:

HackRover is constantly iterating and the direction of the club changes as students come and go. Having extensive, yet concise documentation of what the club is about, a rulebook for the competition related to HackRover, and an onboarding pipeline would solve this pain point.

One Scenario Environment:

The scenario that we want to focus on most is the disaster relief persona. We would be interested in creating robots that are configurable to fit different scenarios and provide first responders with better support models. Socially this would be a positive since in a disaster time is limited and having a robot that was configurable in on short notice to perform a mission would be a social positive. Further, the development context of this device provides a fun educational experience for university students. 

		\subsubsection{User Insight Report}
			After meeting with our sponsor, professor Pierre Mourad, he explained how current engineering firms respond to disaster relief when first responders determine that a specific rover is required. They’d rather send a specifically configured rover into the situation where a responder’s life could be seriously threatened. Disasters happen often and they end up happening fast and have limited time to save as many lives as possible. Rovers would need to be easily configurable, or at least provide enough data under small up-time to legitimize their use in the field. Further, they should be able to direct the efforts of responders as to limit lives lost.

Further discussion with Mourad informed us that we might pivot HackRover to design competitions which reward designs suiting disaster situations by creating simple puzzle elements and point systems which reward the related functions. The puzzles themselves act as a springboard for our design direction. Further, multiple rovers of various configurations will be built simultaneously to see ‘what works’. This will provide first responders with an arsenal of quickly configurable rovers that can be used in any type of disaster scenario, or at least lead to the design of one ‘great’ system spread nebulously in the design of all four systems.

The name of the club/the project would still be HackRover even if the ethical hacking component is not mentioned but the software component would still access the information data of the rover.

	\subsection{Problem Understanding}
		\subsubsection{Product Assumptions}
		\subsubsection{Functional Assumptions}
		Three Needs Statements:
		
- A product that addresses the problem of UW Bothell engineering students being unprepared post-graduation in a competitive job market that creates experienced engineering students from UW Bothell.

- A product that addresses the problem of disaster relief needing specific robots for certain disaster scenarios of a world where in a disaster, every second matters that creates a benefit of saving others and reducing risk for first responders.

- A product that addresses the problem of engineering students being ill-equipped to tackle industry-level challenges within engineering firms that create an extracurricular club attached to schools that provide this industry-level experience. 

		\subsubsection{High Level Usability Constraints}
			Three Needs Statements Successful Variable:

- In this needs statement that tackles the problem of UW Bothell engineering students being unprepared post-graduation, we will know that we were successful by providing a well-rounded education in the club and contacting past HackRover members post-graduation inquiring how HackRover changed their career trajectory.

- We will know that we were successful in this needs statement by creating rovers that are easy to assemble and configure for different disaster tasks. The rovers should be able to complete a set of puzzles that the HackRover competition hosts.

- In this needs statement that tackles the problem of engineering students being ill-equipped to tackle industry-level challenges, we will contact past HackRover members from other universities past graduation inquiring how HackRover changed their career trajectory. 

		\subsubsection{Final Need Statement and User Outcomes}
		A product that addresses the problem of disaster relief needing specific robots for certain disaster scenarios of a world where in a disaster, every second matters that creates a benefit of saving others and reducing risk for first responders.

We made this choice because after speaking with our sponsor, professor Pierre Mourad, he indicated that this need is something that he would like to see come to fruition. The users would want to see a product that is able to be configurable to any specific disaster situation. The major pain points are figuring out what kinds of disaster situations the rover should be equipped with and designing a rover that is easily configurable and accessible by first responders around the world. Our HackRover will be able to solve these pain points by having a wide variety of configurations for any type of disaster. HackRover will also be built around a robust system using the Robotic Operating System (ROS) and creating a template for first responders to utilize and configure to their needs.

		\subsubsection{Hypothesis Statement}
		We believe our users are first responders who need a configurable rover in their arsenal to have a greater response to disasters. We will know that our solution is what they need if we see our rover perform puzzles in tight corridors and be able to have different configurations for a wide variety of puzzles.


		\subsubsection{Possible Inventions and Business Model}
		Possible Inventions

		Current systems for disaster relief are overly-expensive, shallow attempts at a design that jerry-rig existing structures/systems to solve issues of disaster relief. What’s more is that these systems rarely provide meaningful help, have large uptimes, or return limited data back. For example, aerial systems show snapshots from a bird's eye view, but little more; they’re expensive to acquire, time-consuming to deploy, and require a skilled operator. Some might argue that visual feedback is enough to trace access paths and gauge the scope of the damage, but it gives a limited understanding of victim locations, danger points, etc. Another example: most land rovers - despite being capable in their locomotion - return limited useful sensor data (compared to drones), and the current state of actuative technology is limited. The biggest thwart to effective disaster technology, however, is the context in which it’s developed. Frequently made by heavy industry tech giants, the return on investment is minimal and opportunity costs are great; to them, the effort is best served elsewhere. The only place to turn to, then, is University Students who are willing and wanting to work for the low cost of a good resume and contribution to society. Here effective organizations need to be made to mobilize their talents into the creation of disaster relief robots; the best scenario is a robotics competition. Most existing competitions see students participating in events that see them designing rovers tailored to useless/limiting activities, or just working with a confining kit and no creative freedom at all. Though some break the mold, they’re few and far in between, and none focus on disaster relief. To further robotic, disaster relief tech, effective student organizations must be created.
		
	The first obvious ‘invention’ of this capstone is a disaster relief-oriented competition, which we call the HackRover Autonomous Robotics Competition (HARC). Traditionally it existed to provide an outlet for interdisciplinary engineering practice and an exploration of IoT; but, as the competition itself has no context there is a point for improvement. This new iteration emphasizes the creation of puzzle panels with elements and a point system that inspires the design of autonomous systems with capable actuation, computer vision and vision processing, capable locomotion, short up-time, easy configuration, and quick access.

The second main invention is a modular rover framework. In the past, the biggest hurdle for all robotic systems has been the software architecture, which must be uniquely tailored to the hardware being employed. With the help of some middleware Robotic Operating System (ROS), we can create individual software nodes and elements which can be knit together in as little as a day to create unique robotic software stacks. Our goal, then, is to create an easily understandable, well-documented, modular rover framework that supports a near-endless number of arms, locomotive systems (traditional mostly), cameras, and other sensors, limited only by the hardware being used.	

The third invention is a physical disaster relief rover. Founded on the modular rover software framework (discussed as invention two), the design of the relief rover would only concern a small group of electrical and mechanical engineers configuring various pieces of third-party hardware into a physical rover capable of speedy, robust locomotion, payload delivery, actuation, and sensing. All control (arm and locomotion), sensor data processing, and additional actuative capabilities are the subject of the well-designed software framework mentioned.

Our Solution and Invention

The most realistic solution at this point would be the creation/hosting of the HARC at UW Bothell. Though a rover will certainly be built upon which the competition takes place, it will be nowhere near a fully-fleshed, disaster-relief rover desired, but a first iteration. Certainly, many of the components which would show up in a final disaster rover will be present, and an early version of the software framework will be present; but, it can definitely be improved upon for future iterations of the competition.
	
	Despite the apparent shortcomings in the design, the rover will serve as disaster relief in limited contexts and will provide an enriching educational engineering experience for UWB students now and for the future. Below is a representational image of the competition venue Discovery Hall for context. Puzzle elements are laid out on about four floors, accessible by stairs/elevators. Students operate rovers, attempting to solve puzzles and gain points while navigating the building effectively and hacking/slowing other competitors’ rovers. Points are awarded based on solutions to puzzles, specifically designed to encourage rover designs that would be capable in disaster relief contexts.
	
		\subsubsection{Planning Roadmap}
		Initial Product: A competition with a rulebook, at least two rovers (one disaster-relief oriented), and four puzzle panels.
		
		The first competition encourages student involvement, a clear understanding of the framework, and expectations for the iterative methods used in developing our systems. Puzzles won’t be so disaster-tailored as much as they will be a proof of concept. The first two rovers created will be designed to prototype a ‘skeleton’ software framework that exemplifies the principles of the software organization we desire. Physically these rovers will serve different purposes: the first to simply prototype the competition, drive around Discovery Hall, and make use of our available technology, and the second to provide an early model of what a disaster relief rover could be. Focusing more on the latter robotic design, an early disaster relief rover would have quick, rugged locomotion systems, effective sensor data storage, feedback, and/or processing, capable payload delivery, and meaningful arm actuation. At this point in the ‘product’ development, we wouldn’t focus too hard on waterproofing or extremely robust chassis. The obvious adopters here are going to be UWB students, capstone students, Pierre Mourad, STF, and only philanthropic companies.
		
		Product Additions: An improved, R and D-friendly, competition.

	The competition will eventually be formatted to address the desires of tech companies nationwide. Interested investors can trade capital for limited control over the direction of the competition and design requirements of involved rovers. As the competition and participants grow this offer would be more lucrative, and the potential for outsourcing cheap but quality R\& D increased greatly.

		Product Additions: Specialized Software Nodes.
	Mentioned implicitly through the document is the notion that software nodes ‘providing/extending capability’ of the rover may be created. A feature of ROS is that functionality can be compartmentalized to a web-like software architecture. Drastic changes in the hardware do not require extensive recreation of the software; instead, nodes are repurposed and reorganized to achieve a framework that reflects the physical system. A skilled ROS operator with knowledge of the nodes’ functions could create a completely new system within a day, for a totally unique rover. Future product additions could see ‘stacks’ of nodes created for unique disaster relief settings: building collapse, fire, aqueous, etc. Each of these stacks would come with the necessary code to configure context-specific locomotion, sensing, and actuative systems. Instead of purchasing an expensive system that ‘does it all’, clients could buy just one software stack suited to their disaster response expectations. 
	The rovers' modularity will have different packages that a client can use to configure their rover to their specification. For example, if there is a need for a LIDAR system then there is a module that would integrate the LIDAR system for a client. This would already be done physically as a unit that would attach with one connector to the main rover board and the software would just have to have slight code changes to adapt to the new module. 

\pagebreak
		
\section{Experience Design \- Human Interface Design}
	\subsection{User Workflow}
	Before addressing UX in depth, a quick review of the state of the systems. The last capstone sent forth a rover with hardware components ordered according to an (almost) ideal rover architecture, however no software (at least not in ROS) was in place. All control was exclusively done through a script loaded onto the Raspberry Pi (RBPi) which enabled a user to control the rover via a bluetooth game controller. The controls were hardly programmable, and what existed was limited (no control over LiDAR, arm, etc). Before any mind can be paid to serious UX, the software framework has to be created; each node of the rover must be in contact with others via the bluetooth to enable us to create useful control scripts and paradigms. Danny Kha is hard at work on this, and we anticipate early software models which represent a final product by March. 
Despite the little mind paid to UX, we do have some early ideas and expectations. The first two UX choices will be: simple ‘boot up’, and simple control. Scripts in the form of launch files will be created, allowing us to boot multiple nodes and software features of the rover in safe sequence and at the click of a button. Simple control will consist of control scripts (housed on the Xavier) which enable programmable Xbox controllers and keyboard/mouse control. These early controls are expected to be unintuitive, so long as we can map inputs to expected outputs, we can at least rapidly test the function of the rover. 
Long term ideals are a web-based GUI which allows for visual control, easily accessible through a browser. We’d also like to create a node, hopefully with a UI as well, which permits easy control mapping for the XBox and keyboard. Other developments include alternative control schemes: Wii Mote and Nunchuck, Vive Hardware (headgear and controllers), and Switch Controllers; all of these have either open-source hacking communities, or easily accessed programmability. All hardware will come from Justin Heinzig or Pierre Mourad. 
The aforementioned web-based GUI bleeds into our long term ideals. First a remotely accessible GUI via wireless/wired access to the Xavier (though not necessarily through a browser), will need to be created. Additional long term ideals include porting the camera output of LiDAR to a VR headset. 
Shown below is an example of how the HackRover mission control system could be compiled into a user interface for ease of use for the user. The rover operator will have an intuitive interface that enables easy manipulation of the robot along with emergency shutdown and interruptions built into the control. 

	\subsection{Environmental Analysis}
	The environment of the rover is two-fold: competition space with simulated events/circumstance, and confined disaster spaces. The web-based UI and simple wireless control serves to empower competitor’s experience during the event. However, it should also easily map onto the disaster location. Good communication with the rover system as it operates in distant, wireless connected, cramped spaces are a must. Ease of control programming and use is a must. Good sensor and visual feedback is a must. Robust traversing of the environment is a must. 
Assuming all these ‘musts’ mentioned in the last paragraph, the three-pronged solution to satisfy them is: an effective and functional software architecture, a reliable mechanical system, and a long-lasting power distribution system. The software architecture, created in ROS, will be made with modularity and programmability in mind. The mechanical system will transcend its simple motor-shaft design into the realm of geared drive trains and intelligent chassis configuration. The power distribution system will consider as many power-saving mechanisms and electrical stop-gaps to failure as possible. The resulting system should be easy to program, deploy, control, and service.

	\subsection{Cultural Factors Mapping}
	One of the low priority ‘cultural factors’ we’re trying to hit at is in the aesthetic design/flare of our river systems. For each design we provide a meaningful name reflective of the function/shape/aesthetic of the system. Further, we integrate paint or features which are entirely artistic that reflect the name. The goal is to make the design projects fun and enjoyable, and give participating students an end sub-project to look forward to that makes the system they’re developing all the more personal. Flare includes custom, PCB artwork, and painted rover bodies. 
The higher priority - and more meaningful - cultural factor we aim towards is in the control programming of the system. We’re trying to make the rovers reflect commonly understood formats (drive system, actuation, etc) seen in popular science fiction, and real life applications, so understanding how the system interfaces with the world is rapid. We’re also implementing as many video game control paradigms as possible since many of this generation have a familiarity with video games; this eases the operator’s learning curve and provides them the option to control with something they’re familiar with. 
Using an Xbox controller will allow for a natural form of control for the rover. Most of our software will be based around English-speaking audiences but later on can be implemented to add different languages to change the GUI language.


	\subsection{Information Architecture and Hierarchy}
	For startup: Either turning on the rover and running one launch script, or pressing one physical button (attached to the execution of a launch script) will set up the system completely and run all background checks/power up tests. 
For use: Intuitive bluetooth pairing button combinations, or browser-based logins will allow the user to quickly access the rover. The startup launch, mentioned prior, would have enabled the rover to receive simple communication. Perhaps a UI for selecting controls before use will ensure the correct systems are selected. From there all controls should be either intuitive, well documented, visually displayed, or easily accessible from the file system of the rover. 
During use: The user is assumed to be interacting with the rover at this point. All inputs will forward from the control scheme directly to the master node hosted on the Xavier. This outputs control commands to the necessary sections (LiDAR, arm, etc), for which feedback is received. This feedback is processed as we deem fit and then manifest as either actuation of the physical rover or display on a desired monitor. For example, a control input may be processed through some control script, then modified by state conditions of an actuating device to provide some output. Alternatively a change in the environment, or press of a button will alter what sensor information is displayed on some feedback monitor. We plan on having only certain data be displayed in an orderly fashion, rather than providing the user the ability to select what they’d like to see. The reason is that only limited information will be displayed back as it already is, and we’d rather not muddy the use or development process (in creating such a system). Shutdown after use will be the reverse of startup, just as easy to accomplish.

	\subsection{Interface Design}
	The physical components of input are the controller, sensor systems (LiDAR vision, motor position, etc). The physical components of output are a monitor (if it’s needed), the arm, and the motors of the rover. All interfacing is completed by the software on the Nvidia Xavier. 
The user will be able to choose between two different types of analog inputs to control the rover. The first would be the Xbox controller and the second through arrow keys on the GUI either using the keyboard arrow keys or the GUI clickable buttons to command the rover.

Currently we are focusing on integrating the Xbox controller as our best option to control the rover as there are built in functions within Linux that allow joystick control to easily connect with ROS. 


	\subsection{Graphic User Interface Design}
	The graphical user interface hasn’t seriously been considered at this point in the design process. Further, it depends on how much we can ‘hide away’ in the software and implicit controls of the system. Ideally the only UI will be a small status window and output of what the LiDAR sees. Buttons on the controller will let the user switch between LiDAR output modes, (distance maps, raw image, etc). There may also be a visual of the rover with diagnostic points where errors in operation are cast as simple visuals denoting failure. An early example of our Hackrover mission control GUI can be seen below.
	
Some guidelines for the proper use of the Hackrover are as follows:

Rover User - The GUI will allow the user to have full control over the rovers drivetrain and will be able to be easily controlled by users with a controller that is easily deployable. The end user will not be required to have any background knowledge of the rover to operate it as well as knowledge on how to configure the rover itself.

Rover Maintenance - The maintenance of the rover will be accomplished with items that are easily accessible within the rover for replacement and are common to find in the real world.

Autonomous Navigation - The nature of autonomous navigation is to allow the rover to discover its own way around obstacles and reach end points. The best use by the user is to ensure that the end points are reachable by the rover and that the user should assist the rover with finding efficient pathing to reach the end goal. 

	\subsection{Best Use Practices Report}
	Since we haven’t effectively implemented our control systems or completed our design I can’t confidently speak on best use practices. All that can be said about use is contained in B.4. We strive to create a human interface that allows users to be quickly trained on how to operate the rover. With a user interface that has limited items being displayed it allows the user to focus on the core part of controlling the rover rather than trying to figure out what each item being displayed on the GUI represents. 
	
	\subsection{Usability Requirements Document}
	The first product will be a robust rover system with a modular, programmable software architecture created in ROS and hosted on the central Xavier computer. Booting the rover system will be straightforward, and all background tasks for effective startup will be offloaded from the user via automation by simple scripts. The user will be able to easily control the rover from one of many control paradigms, and optionally see output on a monitor. 
	
The form factor of the rover itself will be akin to last year: a PCB interfacing the Xavier with an RBPi and motors; and a LiDAR, arm, and communication stack communicating with the Xavier. All UI will be handled through ROS. 

A further list of pros/cons can’t be provided now since we haven’t finalized our initial design, and haven’t accurately assessed our ability within ROS.

\pagebreak
	
\section{Design}
	\subsection{Decisions and Rationale}
	The final plans for the build phase will - at a minimum - include the creation of one first iteration puzzle panel, one first iteration disaster rover (Mimir), and one first iteration competition rover. Experiencing success with these sooner than later will prompt us to improve upon the base design with further iterations (which introduce more electronics, better mechanical housing, cleaner code), as well as create more puzzle panels and competition rovers.
	
The limiting factor now for mechanical engineers is creation of gear trains. Understanding what goes into drive trains and being able to construct one is necessary. We have a majority of the parts, or have access to creating them with tools/resources in the Disc 270 capstone lab. A good understanding of gear trains will result in 3D prototyping, printing, and testing that may be packaged into test reports placed in this document. Another limiting factor is the surveying of existing mechanical components and their identification with past purchasing sheets and/or engineering professors. When this is done, modeling and simulation will be quick work, and up-time on building the rover should be short.

The limiting factor for electrical engineers is understanding the buck modules, a few electronic components, and some software. We’re taking it one step at a time, determining the behavior of electronic components and figuring how they fit in the larger power distribution systems. Until we have parts selected that can properly step down our voltages as necessary, and deliver stable power to each necessary component, we can’t begin to model the system. With the system modeled we should make quick work of building, as most parts exist.

The limiting factor for computer engineers/scientists is really just understanding the software and operating system we’re working with. We have all the components for the rover, we just have to take them one at a time and integrate them effectively.

	\subsection{Tier 1 Schematics}
	
	\subsection{Tier 2 Schematics}
	Electrical Team (EE Capstone):
	
The puzzle panel schematic that is shown above details the connection of power for the puzzle panels. The connection starts from the power source which is a wall wart that we will use to power the puzzle panel. The wall wart would output its maximum voltage to the power distribution system (PDS) which then will be distributed to every electrical component. The PDS is constructed to be able to provide different ranges of voltage to meet the power requirements for any electronic component. The puzzle panel itself will be operated by Arduino or PCBs which control the function of the puzzle panel through its connection to the electrical components. Each puzzle panel differs by its own use of electrical components and by the function of its Arduino. 
	
	The simplified Rover Electrical Schematic details the connection from the power source to the rest of the electrical components that are present in the system. This figure is bound to change as the system can be reconfigured to fit the physical design of the rover. The entire rover will be powered by a power source that will run each component of the rover. Raspberry Pi, Xavier, and Motor controllers are necessary for the system to control both the distribution of power and as well as sending signals and information to their respective components.  

	
Software Team (CE Capstone):

Figure 1 shown above describes the interaction between the Intel RealSense LIDAR L515 with the NVIDIA Xavier. The ROS node that is used as the publisher of the LIDAR image data to then be sent to a subscriber to the image data uses image\_ transport as the library to export the images from the LIDAR to the processing node in a low-bandwidth compressed format to reduce latency. The image processing node has computer vision models within the node to process the image using OpenCV, which is then sent to the Xavier as data that the Xavier will use for further motion logic or displaying the image to the user.

Controller System Design Diagram:

Figure 2 shown above describes the interaction between controllers (Xbox, Playstation, etc type controllers) with the NVIDIA Xavier. There is the package that contains the motion logic that the previously mentioned LIDAR interacts with to send motion controls to the Xavier as well as a ROS node that takes controller inputs and processes them to be sent to the motion logic package. The data going from the controller to the ROS controller is analog data that Linux can already understand and there are pre-existing ROS libraries to use controllers. The controller node itself will send out motion commands to the motion logic package which will then send motion data to the Xavier. We also want to ensure that the controller node itself won’t do any illegal inputs so we want to create a connection between the motion logic package and the controller node incase of inputs that were not friendly to the rovers motion.	
	
	\subsection{Tier 3 Schematics}
	Electrical Team (EE Capstone):
	
The electrical schematic of the puzzle panel will start with a wall wart which would output a specified amount of voltage to the PDS. The PDS will be made up of three levels, each level containing rows of buck modules that are placed in parallel and their function is to step-down the input voltage from the wall wart, down to the amount that is needed for an electrical component. Each buck module will distribute different amounts of voltage due to each electrical component within the actual puzzle panel, requiring different levels of voltage.

The rover will be powered by lithium batteries that can provide enough voltage or power to the whole rover. The power will be distributed with buck modules to power each component at its needed level. And the step-downed power will then be sent to the Raspberry Pi, Jetson Nano, and Motor controllers which will run the whole rover. A PCB will be constructed to ensure a simple and controlled designed connection of essential components on a single panel.

	Software Team (CE Capstone):
	
Starting with the LIDAR integration the main connection is between the publisher and subscriber node of the LIDAR image. The publisher node will be outputting a compressed formatted image to the subscriber through the image\_ transport API. Within the processing node (subscriber to the image data) we can use OpenCV to translate the image that is being collected from the publisher to be created into a model that the rover can use to identify what is being shown in the image. This can be done with a point-cloud model and computer vision techniques. After the image is processed, the image itself is sent to the Xavier to display to the user and the image data with motion commands is sent to the motion logic, where the motion logic will then determine whether the images motion data is relevant to motion objectives. The latter is a back and forth process between the subscriber processing node and the motion logic component. 
The second part with Xavier for this capstone project is the controller node. The main connection will be between the controller node that takes in controller information and the motion logic. This will also be a back and forth process between the two since the motion logic may not want to perform the actions that the user is inputting into the controller. 



	\subsection{Estimation of Cost of Goods}
	We don’t have a final estimation of the cost of goods. Major design decisions have to be made by electrical and mechanical leads before a final order can be placed. An early order for the first round of parts has already been placed, see below. 


	\subsection{Schedule}
	We make use of an online, free, Gantt software accessible to anyone. A screenshot of only a portion of our schedule can be seen below. Our project is split into mechanical, electrical, and computer sections, each with ‘sub-projects’ which have their own schedule prescribed with dates, information, background, etc.

\pagebreak	

\section{Concept Development}
The electrical system of the rover will consist of many different electrical components that are connected to each other and that can send signal communication to other devices. The system is designed to distribute power to different components from a battery or a single power source.
	
The list of hardware components (bound to change after testing and possible redesigns):

PCB or power distribution system

- The PCB is designed to distribute power to every electrical hardware from a single power source or battery. And as well as, provide the signal connections for the hardware as they are essential in controlling the functionality of every component.

Raspberry Pi

- The Raspberry Pi is a microprocessor that is designed to control the motors by controlling the motor controllers that control the drivetrain.

Jetson Nano

- The Jetson Nano is a small computer that is designed to run multiple neural networks like image processing, object detection, etc. 

LM2596 Buck Converters

- The buck converters are designed to step down or drop down the voltage that is needed by the electrical components to be able to functionally run.

LiDAR camera

- The LiDAR camera is tasked to provide image mapping to the rover.

BTS7960 Motor Controller

- The motor controller is used to receive signals from the Raspberry Pi and delivers it to the actual motors of the drivetrain.

Rover wiring schematic

- The final rover wiring schematic will be provided after tests and final redesigns have been done. 

Risk identification

1. One of the major risks or dangers for the electrical system is the probability of overheating which will cause a fire hazard. This will cause permanent damage to many of the components on the rover. The battery and the components that needed the most power are at risk of this danger, so sufficient protection for these parts will lower the risk of this danger.

2. The risk of the motors running too much or insufficient torque will cause different amounts of currents to be drawn from the battery. This current spike will cause damage to the electrical hardware that is directly connected to the battery and cause a fire hazard. The mechanical components will also face damage as a result.

Software System

The software system of the rover will be centered around using the Robotic Operating System - ROS, which is an open source robotics middleware. The system is designed to live on the NVIDIA Xavier and Raspberry PI on the rover and consists of multiple sections of code that interact within the ROS environment. The software system is used to provide command and control for the rover. The Raspberry PI and NVIDIA Xavier pictures can be seen in Figure D1 and Figure D2 respectively. 

List of ROS packages (bound to change after testing and possible redesigns):

LIDAR Package
- This package consists of a publisher node for the LIDAR image data and two subscriber nodes for outputting the image data as either an imaging processing node or an image viewing node. The image processing node is able to process the image and send corresponding motion data to the main motion logic package. The image viewing node is used to display the image that is coming out of the publisher node to wherever the output is required live.

Controller Package

- This package consists of a publisher node for the controller data and currently one subscriber node that processes the controller data to be sent to the main motion logic package. 

Motion Logic Package

- This package consists of a publisher node that takes motion data from the LIDAR or controller to determine the motion of the rover. The subscriber will be a middle node that connects to the Raspberry PI which is used as the motor controller for the motors on the rover. 

Risk Identification

1) A major risk is the LIDAR node creating a false LIDAR map of the surrounding area and thus sending false information to the motion logic package. We can mitigate this by adding in additional logic to prioritize using controller motion data compared to the LIDAR mapping data. 
2) Another risk is the motion logic not having enough logic within it to ensure that the rover does not make false moves that may endanger the rover itself. This risk can be mitigated through edge case testing and ensuring the motion logic corresponds correctly with rover movements. 
3) The rover system depends on a fully functional user interface for all interactions. If one of the user interface functions does not work correctly, the specific functionality will be affected and not be easily accessed by the user through the application. 


	\subsection{Technology Mapping, Risk Analysis, and Feasibility}
	The minimum product we’ll create is a disaster rover, for which we have all the needed components. Functionality will be ensured with the software created to operate well-planned electromechanical systems. Usability will be ensured with effective documentation and simple controls. A well ordered rover with ease of control and robust build will be necessary and capable for further business; though, our client is the UW, so we have little concern for how this would  sell.
	The disaster rover would need to traverse rugged terrain. A compact frame with a well-set center of gravity and tank treads would allow it to path in any environment. Falls or tumbles would be mitigated by a ‘domed’ upper chassis that ensures it always lands on its base. Operators will be able to connect to it easily and wirelessly, (ideally through browser), and control with easy to understand, video-game based control schemes. All electromechanical systems would interface through robot operating system (ROS) housed on a central, Xavier computer. Focusing on electrical and mechanical systems independently: electronics will convey clean, modular power to various components through LIPO batteries. Clean and modular, I stress again, are the focus; devices like the Xavier have stringent voltage and current regulations. Modularity is desired, as future iterations will see different parts added or subtracted; ideally the only aspects affected here are battery life. Everything is contained on a central PCB, improving physical efficiency further. The mechanical systems need to be robust and sourceable. Through the use of aluminum frame, higher end FDM and resin plastics, and (hopefully) casting, we’ll be able to create a protective housing for all electrical and computer components. Motors will be mounted directly to the frame with little (or no) geared transmission, depending on the speed and torque output of what motors exist. A small frame is preferred to minimize stresses. Further research into the mechanical engineering of motion mechanisms is needed before a better description can be provided. 
	The greatest risks are addressed in order: computer, electrical, mechanical. The computer model only addresses the function of interdependent components in extremely ideal conditions. A disaster environment  won’t be perfect, and the rover needs to get itself out of sticky situations if communication is lost or electromechanical systems aren’t behaving perfectly. The electrical system could fail to provide stable enough current or voltage, or locales of high power draw could result in heating that leads to failure. It’s possible the arrangement of components doesn’t permit a long enough battery, or results in excess inefficiencies. The mechanical systems may fail about shafts or joints, and the production of gears (as they’ll be printed) may be outside of tolerance.
	To address the stated risks we need effective testing and redundant systems. Computer models need to simulate circumstances where power delivery or mechanical failure causes issues. Further, boot scripts and backup communication protocols need to exist. Electrical systems have to be tested under every conceivable driven condition, with a focus on clean power. Mechanical systems need to be simulated, constructed with high factors of safety, and modeled physically with test rigs.
 	 

	\subsection{Design Strategy, Concept Development, and Concept Risk Identification \& Mitigation}


		\subsubsection{Technology Mapping, Component Technology Research}
		
		The Project GANTT chart can be added here for the technology mapping image.

		\subsubsection{Electrical}
		The electrical circuit or model will be tested with the mechanical models, and the approximate date of testing will take place from the end of April to May. Testing will involve making sure connections are stable and all electrical components are functioning correctly and correspondingly communicate with each other. 
		
Future testing of electrical components and their functions:
- The connections between components
- PCB functionality
- Nano connection
- Raspberry Pi and Drivetrain connection
- Battery and power connection
The main risk of the electrical circuit or model is that the electrical connection between any electrical components has failed. The risk mitigation can involve:
1.Rewiring/replacing wires, checking every connection between every electrical component.
2.Replacing old/faulty components with new ones can decrease the risk of circuit failure.
Another risk of the electrical circuit is the uneven distribution of power to any electrical component through voltage/current spikes and voltage/current noise. The risk mitigation can involve:
1. Adding new electrical components that are used to mitigate or decrease voltage/current spikes.
- Snubber diodes, bypass capacitors, transistors, and any other component that can prevent undesirable effects. 
2. Adding fuses can be used as a fail-safe to prevent the failure of the entire electrical system.

Future risks will be identified with further testing of the models.

		\subsubsection{Mechanical}
		\subsubsection{Firmware and Software}
		The rover system will be tested within computer simulation and with the rover itself. The testing will take place between April to May and will test the components that were created such as the LIDAR, controller, and motion logic packages. We will also test that the interaction between all the packages are working as expected. 
- LIDAR system testing will consist of ensuring that the image display is displayed correctly on the outputting display and that the image processing node is providing motion logic data that is accurate to the map that the LIDAR creates.
- Controller system testing will consist of input testing of single or multiple input prompts and ensuring that the rover can respond to the motion logic that comes out of the controller system. 
- Motion logic system testing will coincide with controller system testing as we need inputs to ensure that the motion logic is moving the right motors and the connection between the Xavier and PI are correct. In this we will validate that the motion logic system can turn the correct motor on command and can work independently from the LIDAR or controller if we do manual inputs through the console. To reduce the risk about the incorrect movements from the controller to the motion logic, we will run edge cases with a multitude of inputs to the controller and observe how the motion logic responds to them.

		\subsubsection{Human Interaction Models}
		The main interaction between the user and the prototype will be through a PC and possibly an XBOX controller which would control the rover. The PC will allow the user to start the rover through the ROS language and be able to test the prototype. The XBOX controller would be used to navigate the rover with joint sticks to do basic operations. And this controller works best for the user to control the rover and wouldn’t require them to know ROS. The risk mitigation involves: 
1. Allowing the user to use the software without first-hand knowledge and being able to test the prototype with ease. This would also allow easy control and having it be user-friendly.

		\subsubsection{System Interface Modeling, Testing, and Risk Analysis}
Testing the system interface of the rover, we would ensure that the connections between the interface are functioning accurately. Some of the components which we would be testing are the connection for the LiDAR camera. 

The main risk of the system interface would be the failure in the connection between the devices. Mitigating the risk would involve:

1. Testing each connection between systems and their interface functionality.
2. Testing the LiDAR camera to ensure that the mapping is accurate and readable. 

The testing of these systems will be finished around May and more progress will be done as the team continues to design and build the actual systems.

	
\end{document}
