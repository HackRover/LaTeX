\documentclass[a4paper, 10pt]{article}

%Replace title
\title{Puzzle Panel Iteration 1 Standard Notes}
\date{\today}

\begin{document}
\maketitle

\pagebreak

\tableofcontents

\pagebreak

\section{Power Distribution System (PDS)}
A power distribution system will be an arrangement of electronic components capable of taking power from a wall-wart and transforming it into easy-to-use, clean energy suppliable to a variety of electromechanical puzzle components. The current design idea: multiple, adjustable buck modules arranged in parallel, accepting power in from the wall wart, and 'bucking' it down to 5V (or whatever we need). This new power will be distributed in parallel to female pin headers. Students can then reliably plug in Arduino components to the headers without worry of overpowering or destroying their devices. All of these electronic components will be contained on GikFun solderable breadboards. Future iterations will see a printed circuit board.

	\subsection{NO DATE: Catchup}
		\subsubsection{Summary}
		We didn't have these LaTeX templates done in time so I'm creating a 'catchup section' to catalogue our progress, and recount some of the roadblocks I remember us facing.
		
		We created the first reliable iteration of the PDS. A 3D modelled, and printed housing affixes with metric screws to the backside of the puzzle panel. The soldered breadboards with all components slot onto rails of the PDS housing; only female pin headers sit exposed to allow students to plug in Arduino components. Everything appears to be in working order and it will soon be affixed too the back of the panel, ready for further testing.
			
			\subsubsection{Roadblock: Unknown Buck Modules}
			We used buck modules from the cabinet, but we weren't aware of their history. Mario Ge tested all the buckmodules by running some voltage and current through them on a benchtop power supply module. Many were non-functional, so he disassembled and saved what he could of use, (potentiometers, capacitors, etc). Those that worked were bagged for future use on the PDS.
			
			\subsubsection{Roadblock: Questionable Buck Module Function}
			The biggest worry was that the buck modules had no ability to control power distribution the way we wanted. Would the module deliver as much current as possible, or would it only let components pull as much as they needed? We want devices to be supplied with only what they need, if the buck module didn't limit current or simply convey what the part needed things would blow up and not work. 
			
			We performed some simple tests and determined the buck modules output their set voltage precisely (as we expected) but also that they delivered a minimum current which was appreciable smaller than the devices we'll use. Further, the device - not the module - pulls current from the wall wart hooked into the building. This means a motor, light, or sensor will only draw as much current as it needs, no more, no less. This is good, now the only minor concern is that the devices we place don't overwhelm the PDS.
			
			\subsubsection{Roadblock: Building the Physical Model}
			There was no concern for the PDS housing, it was completed by Justin Heinzig in the matter of an hour, and printed fine for the most part (aside from some minor printer-specific fails). The main concern was the soldering and wiring arranngement, as the buck modules didn't fit on any GikFun board. We found snapping one board in half and placing it on either side of a normal board gave us the real estate needed. The final product came out nicely. 
			
		\subsection{Reference/Resources}
		There weren't any noteworthy resources to list here. All of our knowledge came from our degree or past off-the-cuff familiarity with the technology. You could try looking up a video on buck modules. Anthony Harowitz's book, \emph{The Art of Electronics} is also a great resource, sure to have further details.
		
\section{Puzzle Panel Construction}
	\subsection{NO DATE: Catchup}
		\subsubsection{Summary}
		The puzzle panel was first modelled in Fusion360, constructed out of 2x4's and plywood. A physical model was created 4/3/23 - with a few fasteners not seen in the Fusion360 model. 
		
		\subsubsection{Roadblock: Construction Challenges}
		The order of construction could be better considered; weird angles and dangerous positions had to be used to ensure things went together. 
		
		A cutting technique was learned: place plywood on styrofoam insulation then set the circular saw blade depth to just the surface of the insulation. The saw cuts through the wood and just knicks the surface of the insulation; this technique works better than saw horses and lets you work on the ground.
		
	\subsection{4/3/23}
		\subsubsection{Summary}
		Further consntruction continued. An access hole was cut out with a handheld router and velcro was attached. I found that for the velcro purchased, one role of sharp velcro covers 2/3 the board perfectly, as it's face dimensions are 18'' by 18'' and the velcro is 36'' in length.
		
\section{Puzzle Component: Light Strips}
	\subsection{NO DATE: Catchup}
		\subsubsection{Summary}
		The light strips are a visual aid affixed to the puzzle panel. They react when the robot interacts with puzzle elements, as the puzzle designer intends. They're powered via 5V source from the panel's PDS, but controlled by the Arduino. The OEM product is a cuttable light strip which is modified by adding velcro backing and a white painters tape to diffuse the light. At the cut ends, exposed copper must be modified with accessible pin headers.

\section{Puzzle Component: Button}
	\subsection{NO DATE: Catchup}
		\subsubsection{Summary}
		The button is a puzzle element consisting of a 3D printed housing, simple GikFun button element, an LED light, and some accessory electronics for function. The housing presses and/or tabs together, encasing the stock button and LED. It's face not only acts as a large interaction point for the rover, but diffuses the LED's light. Wires solder to pin endings and trail to female headers which stick out of the button as access pis: GND, power, and LED. 
		\subsubsection{Roadblock: Modular Design}
		Two aspects of modular design desired early on were: (1) ability to switch between a toggle and press style button, and (2) ability to opt-out of using the LED. The first we figured could be solved in code, as opposed to hardware. Instead of making complex hardware modifications, the user will have to record button state, (if they choose to employ a toggle button). The second was solved by simply making the button and LED in parallel, while connecting to a common GND. If the user never inserts any pins into the LED port, the button will continue to function.

\section{Puzzle Component: Potentiometer}
	\subsection{NO DATE: Catchup}
		\subsubsection{Summary}
		The potentiometer is just as the name suggests: a housed potentiometer. The housing allows for ease of interactionn by the rover. 
		\subsubsection{Roadblock: Variety of Pots}
		The pots we have on store have questionable data sheets, and their resistance values have extreme ranges and horrid tolerances. This results in either electronic or software exceptions needing to be made depending on the resistor in use. \textbf{This problem hasn't yet been totally solved, so future editors should address it with the same title}; but, we surmise it can be fixed with the Arduino's built in map function. Perhaps a variable be added to the class declaration of \textit{OUR} potentiometer library wherein the resistance value of the pot is added and the computer deals with properly mapping to voltage.
		
\section{Puzzle Component: Ultrasound Sensor}
	\subsection{NO DATE: Catchup}
		\subsubsection{Summary}
		The ultrasound sensor measures distance via an ultrasound pulse. We use a standard Elegoo ultrasound sensor and a housing is developed for it in Fusion360. One important note about the housing is it naturally inclines the sensor at 2 degrees so it won't trip on the mount or board. Another, the housing for the device - as of 4/3/23 -  doesn't have any attachements for the servo motor, though this needs to be added before the end of iteration 1.

\section{Puzzle Component: Servo}
	\subsection{NO DATE: Catchup}
		\subsubsection{Summary}
		Only the plans for the servo have been imagined, little has been done in the way of design; except for the software, for which appreciable progress has been made.  The servo is just a stock servo compatible with the Arduino, a housing should eventually be created, mostly for aesthetics but also for mounting. The servo actuates components on the puzzle panel.
		 
\section{Software Libraries}
		
\end{document}